\documentclass[english, fontsize=12pt, paper=a4, twoside=false, open=right, draft=true, pagesize=auto, version=last, DIV=16]{scrartcl}


\let\counterwithout\relax
\let\counterwithin\relax

\usepackage[utf8]{inputenc}
%Für Tabellen
\usepackage{tabularx}

% Für Absätze in Bildbeschreibung
\usepackage{caption}

% Zur nummerierten Aufzählung, mit automatischem Einrücken
\usepackage{paralist}
\usepackage{enumitem}



% mitwachsender Implikationspfeil mit Text
\makeatletter
\newcommand{\xRightarrow}[2][]{\ext@arrow 0359\Rightarrowfill@{#1}{#2}}
\makeatother


% Java Code Einbinden
\usepackage{listings}
\usepackage{color}  
\usepackage[svgnames]{xcolor} 



% Um einzelne Seiten zu drehen
\usepackage{pdflscape}
%\usepackage{rotating}
%\usepackage{lscape}


\usepackage{lmodern}
\usepackage[T1]{fontenc}
%\usepackage[latin1]{inputenc}
\usepackage{babel}
\usepackage[utf8]{inputenc}
\usepackage{stmaryrd}
\usepackage{extarrows} 
\usepackage{ulem}



%Zur Bildnummerierung
\usepackage{chngcntr}
\usepackage{mathtools}
\usepackage{amsmath}		
%Zur Verwendung von "chngcntr": \counterwithin{figure}{section}


%Euro-Zeichen
\usepackage{eurosym}

\usepackage{float}

\usepackage{amsmath}
\usepackage{acronym} %[''Optionen'']
\usepackage{esvect}  %Für Vektorpfeile, Aufruf mit \vv{Vektorname}

% Für schön Buchstaben1
\usepackage{mathrsfs}
\usepackage{suetterl}
\usepackage{calligra}

%%% Neue Kommandos/Begriffe %%%
%\renewcommand{\thesection}{\arabic{section}}
% Für Fußnoten ohne Nummer
%\renewcommand{\thefootnote}{}



% Für die Normalschrift im Abkürzungsverzeichnis, das Paket "acronym" veranlasst eine andere Schriftart bei Abkürzungen.
\newcommand{\Rule}{\rule{\textwidth}{0.5mm}}
% \newcommand{\absatzParagraph}[1]{\paragraph{#1}\mbox{}\\}


\usepackage{geometry}
\newgeometry{left=18mm,right=18mm,top=15mm,bottom=15mm}
\footskip = 22pt

\usepackage{setspace}  % Paket einbinden
\onehalfspacing        % einstellen des Zeilenabstandes auf 1,5
\setlength{\parindent}{0in}
\usepackage{amsmath}
\usepackage[amsmath,amsthm,thmmarks]{ntheorem}
\usepackage{amssymb}


% Komplexitätsklassen
% \usepackage[bold,full]{complexity}


% Für Pseudocode
\usepackage{algorithm}
\usepackage{algpseudocode}


% für mehrzeilige Kommentare
\usepackage{verbatim}


% ------------- Beginn Definition von: Satz, Lemma Definition usw. -------------
\theoremstyle{break}
%\theoremstyle{definition}
\theorembodyfont{\upshape}
\newtheorem{defin}{Definition}[section] % Präambel
\newtheorem{lemma}[defin]{Lemma}
\newtheorem{satz}[defin]{Satz}
\newtheorem{kor}[defin]{Korollar}
\newtheorem{beo}[defin]{Beobachtung}
\newtheorem{bemerk}[defin]{Bemerkung}
\newtheorem{übung}[defin]{Übung}

%  ------------- Ende Definition von: Satz, Lemma Definition usw. -------------

% Für URLs
\usepackage{url}
\usepackage{hyperref}
\hypersetup{final}



\usepackage{animate}
\usepackage{graphicx}
\usepackage{graphics}


\usepackage{tikz}
\usepackage{tkz-euclide}
%\usepackage{tikz,fullpage}
\usetikzlibrary{calc,patterns,through,intersections}
%\usepackage{tkz-berge}



\begin{document}


\title{
\vspace*{-10mm}
Exercise 2 \\[-3pt]
{\Large $\mathrm{for \ the \ lecture}$} \\[-3pt]
{\LARGE \textbf{Computational Geometry}}
}
\author{Dominik Bendle, Stefan Fritsch, Marcel Rogge and Matthias Tschöpe}
\maketitle
\vspace*{-10mm}

\section*{Exercise 1 (Point between Line Segments)}
Given a balanced binary tree which lexicographically stores a tuple $(x_{hi}, x_{lo})$ for each line segment, where $x_{hi}$ and $x_{lo}$ are the x coordinate for the upper and lower end points.\\
Find the two disjunct line segments that define the region in which the query point $p \in D$ is:\\

node = tree.root\\
left\_line\_segment = null\\
right\_line\_segment = null\\

for i = 0;  i < tree.height; i++\\
\hspace*{10mm}if node.$x_{hi} < p_x$ \&\& node.$x_{lo} < p_x$\\
\hspace*{20mm}left\_line\_segment = node\\
\hspace*{20mm}node = node.right\_child\\
\hspace*{10mm}else if node.$x_{hi} >= p_x$ \&\& node.$x_{lo} >= p_x$\\
\hspace*{20mm}right\_line\_segment = node\\
\hspace*{20mm}node = node.left\_child\\
\hspace*{10mm}else\\
\hspace*{20mm}position = $(x_{lo} - x_{hi}) * (p_y - 1) - (-1) * (p_x - x_{hi})$\\
\hspace*{20mm}if position > 0\\
\hspace*{30mm}left\_line\_segment = node\\
\hspace*{30mm}node = node.right\_child\\
\hspace*{20mm}else\\
\hspace*{30mm}right\_line\_segment = node\\
\hspace*{30mm}node = node.left\_child\\

Query point p is in the region between$^1$ the two lines segments:\\
$\overline{(left\_line\_segment.x_{hi}, 1)\hspace*{3mm}(left\_line\_segment.x_{lo}, 0)}$\\
$\overline{(right\_line\_segment.x_{hi}, 1)\hspace*{3mm}(right\_line\_segment.x_{lo}, 0)}$\\

$^1$If a point lies exactly on a line, it is treated as if the point lies in the region to the left of that line.\\

The algorithm does not work for arbitrary line segments.\\
Arbitrary line segments can have intersections which causes multiple problems. For example could the two line segments, that define the region in which p lies, be on different branches. This would require to traverse the tree two times. At the same time, could the region, in which the point p lies, be defined by more than two line segments. This would mean that the tree would then have to be traversed an unknown amount of times.
\newpage




\section*{Exercise 2 (Line Segment Intersection)}
$Q = tree(p_1,p_2,p_3,q_1,q_3,p_4,q_2,p_5,q_5,q_4); \ T = tree()$ \par
\vspace*{-3mm}
\hrulefill \\
$p = p_1, U = \{s_1\}, L=\emptyset, C=\emptyset$ \\
$|L \cup U \cup C|>1 \text{ is false}$ \\
$\text{no deletion, }T.insert(s1) \Rightarrow T=tree(s_1)$ \\
$U \cup C=\{s_1\}$ \\
$s'=s_1, s_l=null$ \\ 
{\scshape{FindNewEvent}}$(s_l,s',p) \Rightarrow$ $s_l$ and $s'$ do not intersect below the current sweep line. \\
$s''=s_1, s_r=null$ \\ 
{\scshape{FindNewEvent}}$(s'',s_r,p) \Rightarrow$ $s''$ and $s_r$ do not intersect below the current sweep line. \\
$\hspace*{49.5mm} \Rightarrow Q=tree(p_2,p_3,q_1,q_3,p_4,q_2,p_5,q_5,q_4)$ \\
intersection\_points $= \emptyset$ \par
\vspace*{-3mm}
\hrulefill \\
$p = p_2, U = \{s_2\}, L=\emptyset, C=\emptyset$ \\
$|L \cup U \cup C|>1 \text{ is false}$ \\
$\text{no deletion, }T.insert(s_2) \Rightarrow T=tree(s_2,s_1)$ \\
$U \cup C=\{s_2\}$ \\
$s'=s_2, s_l=null$ \\ 
{\scshape{FindNewEvent}}$(s_l,s',p) \Rightarrow$ $s_l$ and $s'$ do not intersect below the current sweep line. \\
$s''=s_2, s_r=s_1$ \\ 
{\scshape{FindNewEvent}}$(s'',s_r,p) \Rightarrow$ $s''$ and $s_r$ intersect below the current sweep line in $p_{s_1,s_2}$. \\
$\hspace*{49.5mm} \Rightarrow Q.insert(p_{s_1,s_2})$ \\
$\hspace*{49.5mm} \Rightarrow Q=tree(p_{s_1,s_2},p_3,q_1,q_3,p_4,q_2,p_5,q_5,q_4)$ \\
intersection\_points $= \emptyset$ \par
\vspace*{-3mm}
\hrulefill \\
$p = p_{s_1,s_2}, U = \emptyset, L=\emptyset, C=\{s_1,s_2\}$ \\
$|L \cup U \cup C|>1 \text{ is true} \Rightarrow $ $p = p_{s_1,s_2}$ is intersection of $s_1$ and $s_2$ \\
$T.delete(s_1), \ T.delete(s_2), \ T.insert(s_1), \ T.insert(s_2) \ \Rightarrow T=tree(s_1,s_2)$ \\
$U \cup C=\{s_1, s_2\}$ \\
$s'=s_1, s_l=null$ \\ 
{\scshape{FindNewEvent}}$(s_l,s',p) \Rightarrow$ $s_l$ and $s'$ do not intersect below the current sweep line. \\
$s''=s_2, s_r=null$ \\ 
{\scshape{FindNewEvent}}$(s'',s_r,p) \Rightarrow$ $s''$ and $s_r$ do not intersect below the current sweep line. \\
$\hspace*{49.5mm} \Rightarrow Q=tree(p_3,q_1,q_3,p_4,q_2,p_5,q_5,q_4)$ \\
intersection\_points $= \{p_{s_1,s_2}\}$ \par
\vspace*{-3mm}
\hrulefill \\
\newpage
$p = p_3, U = \{s_3\}, L=\emptyset, C=\emptyset$ \\
$|L \cup U \cup C|>1 \text{ is false}$ \\
$\text{no deletion, }T.insert(s_3) \Rightarrow T=tree(s_3,s_1,s_2)$ \\
$U \cup C=\{s_3\}$ \\
$s'=s_3, s_l=null$ \\ 
{\scshape{FindNewEvent}}$(s_l,s',p) \Rightarrow$ $s_l$ and $s'$ do not intersect below the current sweep line. \\
$s''=s_3, s_r=s_2$ \\ 
{\scshape{FindNewEvent}}$(s'',s_r,p) \Rightarrow$ $s''$ and $s_r$ do not intersect below the current sweep line. \\
$\hspace*{49.5mm} \Rightarrow Q=tree(q_1,q_3,p_4,q_2,p_5,q_5,q_4)$ \\
intersection\_points $= \{p_{s_1,s_2}\}$ \par
\vspace*{-3mm}
\hrulefill \\
$p = q_1=q_3, U = \emptyset, L=\{s_1,s_3\}, C=\emptyset$ \\
$|L \cup U \cup C|>1 \text{ is true} \Rightarrow $ $p = q_1=q_3$ is intersection of $s_1$ and $s_3$ \\
$T.delete(s_1), T.delete(s_3) \text{, no insertion} \Rightarrow T=tree(s_2)$ \\
$U \cup C=\emptyset$ \\
$s_l=null, s_r=null$ \\ 
{\scshape{FindNewEvent}}$(s_l,s_r,p) \Rightarrow$ $s_l$ and $s_r$ do not intersect below the current sweep line. \\
$\hspace*{49.5mm} \Rightarrow Q=tree(p_4,q_2,p_5,q_5,q_4)$ \\
intersection\_points $= \{p_{s_1,s_2},q_1=q_3\}$ \par
\vspace*{-3mm}
\hrulefill \\
$p = p_4, U = \{s_4\}, L=\emptyset, C=\emptyset$ \\
$|L \cup U \cup C|>1 \text{ is false}$ \\
$\text{no deletion}, \ T.insert(s_4) \Rightarrow T=tree(s_4,s_2)$ \\
$U \cup C=\{s_4\}$ \\
$s'=s_4, s_l=null$ \\ 
{\scshape{FindNewEvent}}$(s_l,s',p) \Rightarrow$ $s_l$ and $s'$ do not intersect below the current sweep line. \\
$s''=s_4, s_r=s_2$ \\ 
{\scshape{FindNewEvent}}$(s'',s_r,p) \Rightarrow$ $s''$ and $s_r$ do not intersect below the current sweep line. \\
$\hspace*{49.5mm} \Rightarrow Q=tree(q_2,p_5,q_5,q_4)$ \\
intersection\_points $= \{p_{s_1,s_2},q_1=q_3\}$ \par
\vspace*{-3mm}
\hrulefill \\
$p = q_2=p_5, U = \{s_5\}, L=\{s_2\}, C=\emptyset$ \\
$|L \cup U \cup C|>1 \text{ is true} \Rightarrow $ $p = q_2=p_5$ is intersection of $s_2$ and $s_5$ \\
$T.delete(s_2), T.insert(s_5) \Rightarrow T=tree(s_4,s_5)$ \\
$U \cup C=\{s_5\}$ \\
$s'=s_5, s_l=s_4$ \\ 
{\scshape{FindNewEvent}}$(s_l,s',p) \Rightarrow$ $s_l$ and $s'$ do not intersect below the current sweep line. \\
$s''=s_5, s_r=null$ \\ 
{\scshape{FindNewEvent}}$(s'',s_r,p) \Rightarrow$ $s''$ and $s_r$ do not intersect below the current sweep line. \\
$\hspace*{49.5mm} \Rightarrow Q=tree(q_5,q_4)$ \\
intersection\_points $= \{p_{s_1,s_2},q_1=q_3,q_2=p_5\}$ \par
\vspace*{-3mm}
\hrulefill \\
\newpage
$p = q_5, U = \emptyset, L=\{s_5\}, C=\emptyset$ \\
$|L \cup U \cup C|>1 \text{ is false}$ \\
$T.delete(s_5), \text{no insertion} \Rightarrow T=tree(s_4)$ \\
$U \cup C=\emptyset$ \\
$s_l=null, s_r=null$ \\ 
{\scshape{FindNewEvent}}$(s_l,s_r,p) \Rightarrow$ $s_l$ and $s_r$ do not intersect below the current sweep line. \\
$\hspace*{49.5mm} \Rightarrow Q=tree(q_4)$ \\
intersection\_points $= \{p_{s_1,s_2},q_1=q_3,q_2=p_5\}$ \par
\vspace*{-3mm}
\hrulefill \\
$p = q_4, U = \emptyset, L=\{s_4\}, C=\emptyset$ \\
$|L \cup U \cup C|>1 \text{ is false}$ \\
$T.delete(s_4), \text{no insertion} \Rightarrow T=tree()$ \\
$U \cup C=\emptyset$ \\
$s_l=null, s_r=null$ \\ 
{\scshape{FindNewEvent}}$(s_l,s_r,p) \Rightarrow$ $s_l$ and $s_r$ do not intersect below the current sweep line. \\
$\hspace*{49.5mm} \Rightarrow Q=tree()$ \\
intersection\_points $= \{p_{s_1,s_2},q_1=q_3,q_2=p_5\}$ \par
\vspace*{-3mm}
\hrulefill \\
\newpage



\section*{Exercise 3 (Pyramids Skyline)}
\newpage


\end{document}



