\documentclass[english, fontsize=12pt, paper=a4, twoside=false, draft=true, pagesize=auto, version=last, DIV=16]{scrartcl}


\let\counterwithout\relax
\let\counterwithin\relax

\usepackage[utf8]{inputenc}
%Für Tabellen
\usepackage{tabularx}

% Für Absätze in Bildbeschreibung
\usepackage{caption}

% Zur nummerierten Aufzählung, mit automatischem Einrücken
\usepackage{paralist}
\usepackage{enumitem}
\usepackage{csquotes}


% mitwachsender Implikationspfeil mit Text
\makeatletter
\newcommand{\xRightarrow}[2][]{\ext@arrow 0359\Rightarrowfill@{#1}{#2}}
\makeatother


% Java Code Einbinden
\usepackage{listings}
\usepackage{color}
\usepackage[svgnames]{xcolor}



% Um einzelne Seiten zu drehen
\usepackage{pdflscape}
%\usepackage{rotating}
%\usepackage{lscape}


\usepackage{lmodern}
\usepackage[T1]{fontenc}
%\usepackage[latin1]{inputenc}
\usepackage{babel}
\usepackage[utf8]{inputenc}
\usepackage{stmaryrd}
\usepackage{extarrows}
\usepackage{ulem}



%Zur Bildnummerierung
\usepackage{chngcntr}
\usepackage{mathtools}
\usepackage{amsmath}
%Zur Verwendung von "chngcntr": \counterwithin{figure}{section}


%Euro-Zeichen
\usepackage{eurosym}

\usepackage{float}

\usepackage{amsmath}
\usepackage{acronym} %[''Optionen'']
\usepackage{esvect}  %Für Vektorpfeile, Aufruf mit \vv{Vektorname}

% Für schön Buchstaben1
\usepackage{mathrsfs}
\usepackage{suetterl}
\usepackage{calligra}

%%% Neue Kommandos/Begriffe %%%
%\renewcommand{\thesection}{\arabic{section}}
% Für Fußnoten ohne Nummer
%\renewcommand{\thefootnote}{}



% Für die Normalschrift im Abkürzungsverzeichnis, das Paket "acronym" veranlasst eine andere Schriftart bei Abkürzungen.
\newcommand{\Rule}{\rule{\textwidth}{0.5mm}}
% \newcommand{\absatzParagraph}[1]{\paragraph{#1}\mbox{}\\}


\usepackage{geometry}
\newgeometry{left=18mm,right=18mm,top=15mm,bottom=15mm}
\footskip = 22pt

\usepackage{setspace}  % Paket einbinden
\onehalfspacing        % einstellen des Zeilenabstandes auf 1,5
\setlength{\parindent}{0in}
\usepackage{amsmath}
\usepackage[amsmath,amsthm,thmmarks]{ntheorem}
\usepackage{amssymb}


% Komplexitätsklassen
% \usepackage[bold,full]{complexity}


% Für Pseudocode
\usepackage{algorithm}
\usepackage{algpseudocode}


% für mehrzeilige Kommentare
\usepackage{verbatim}


% ------------- Beginn Definition von: Satz, Lemma Definition usw. -------------
\theoremstyle{break}
%\theoremstyle{definition}
\theorembodyfont{\upshape}
\newtheorem{defin}{Definition}[section] % Präambel
\newtheorem{lemma}[defin]{Lemma}
\newtheorem{satz}[defin]{Satz}
\newtheorem{kor}[defin]{Korollar}
\newtheorem{beo}[defin]{Beobachtung}
\newtheorem{bemerk}[defin]{Bemerkung}
\newtheorem{übung}[defin]{Übung}

%  ------------- Ende Definition von: Satz, Lemma Definition usw. -------------

% Für URLs
\usepackage{url}
\usepackage{hyperref}
\hypersetup{final}



\usepackage{animate}
\usepackage{graphicx}
\usepackage{graphics}


\usepackage{tikz}
\usepackage{tkz-euclide}
%\usepackage{tikz,fullpage}
\usetikzlibrary{calc,patterns,through,intersections}
%\usepackage{tkz-berge}

\newcommand*\bigO{\mathcal{O}}



\begin{document}


\title{
\vspace*{-10mm}
Exercise 2 \\[-3pt]
{\Large $\mathrm{for \ the \ lecture}$} \\[-3pt]
{\LARGE \textbf{Computational Geometry}}
}
\author{Dominik Bendle, Stefan Fritsch, Marcel Rogge and Matthias Tschöpe}
\maketitle
\vspace*{-10mm}

\section*{Exercise 1 (Point between Line Segments)}
Given a balanced binary tree which lexicographically stores a tuple $(x_{hi}, x_{lo})$ for each line segment, where $x_{hi}$ and $x_{lo}$ are the x coordinate for the upper and lower end points.\\
Find the two disjunct line segments that define the region in which the query point $p \in D$ is:\\

node = tree.root\\
left\_line\_segment = null\\
right\_line\_segment = null\\

for i = 0;  i < tree.height; i++\\
\hspace*{10mm}if node.$x_{hi} < p_x$ \&\& node.$x_{lo} < p_x$\\
\hspace*{20mm}left\_line\_segment = node\\
\hspace*{20mm}node = node.right\_child\\
\hspace*{10mm}else if node.$x_{hi} >= p_x$ \&\& node.$x_{lo} >= p_x$\\
\hspace*{20mm}right\_line\_segment = node\\
\hspace*{20mm}node = node.left\_child\\
\hspace*{10mm}else\\
\hspace*{20mm}position = $(x_{lo} - x_{hi}) * (p_y - 1) - (-1) * (p_x - x_{hi})$\\
\hspace*{20mm}if position > 0\\
\hspace*{30mm}left\_line\_segment = node\\
\hspace*{30mm}node = node.right\_child\\
\hspace*{20mm}else\\
\hspace*{30mm}right\_line\_segment = node\\
\hspace*{30mm}node = node.left\_child\\

Query point p is in the region between$^1$ the two lines segments:\\
$\overline{(left\_line\_segment.x_{hi}, 1)\hspace*{3mm}(left\_line\_segment.x_{lo}, 0)}$\\
$\overline{(right\_line\_segment.x_{hi}, 1)\hspace*{3mm}(right\_line\_segment.x_{lo}, 0)}$\\

$^1$If a point lies exactly on a line, it is treated as if the point lies in the region to the left of that line.\\

The algorithm does not work for arbitrary line segments.\\
Arbitrary line segments can have intersections which causes multiple problems. For example could the two line segments, that define the region in which p lies, be on different branches. This would require to traverse the tree two times. At the same time, could the region, in which the point p lies, be defined by more than two line segments. This would mean that the tree would then have to be traversed an unknown amount of times.
\newpage




\section*{Exercise 2 (Line Segment Intersection)}
$Q = tree(p_1,p_2,p_3,q_1,q_3,p_4,q_2,p_5,q_5,q_4); \ T = tree()$ \par
\vspace*{-3mm}
\hrulefill \\
$p = p_1, U = \{s_1\}, L=\emptyset, C=\emptyset$ \\
$|L \cup U \cup C|>1 \text{ is false}$ \\
$\text{no deletion, }T.insert(s1) \Rightarrow T=tree(s_1)$ \\
$U \cup C=\{s_1\}$ \\
$s'=s_1, s_l=null$ \\
{\scshape{FindNewEvent}}$(s_l,s',p) \Rightarrow$ $s_l$ and $s'$ do not intersect below the current sweep line. \\
$s''=s_1, s_r=null$ \\
{\scshape{FindNewEvent}}$(s'',s_r,p) \Rightarrow$ $s''$ and $s_r$ do not intersect below the current sweep line. \\
$\hspace*{49.5mm} \Rightarrow Q=tree(p_2,p_3,q_1,q_3,p_4,q_2,p_5,q_5,q_4)$ \\
intersection\_points $= \emptyset$ \par
\vspace*{-3mm}
\hrulefill \\
$p = p_2, U = \{s_2\}, L=\emptyset, C=\emptyset$ \\
$|L \cup U \cup C|>1 \text{ is false}$ \\
$\text{no deletion, }T.insert(s_2) \Rightarrow T=tree(s_2,s_1)$ \\
$U \cup C=\{s_2\}$ \\
$s'=s_2, s_l=null$ \\
{\scshape{FindNewEvent}}$(s_l,s',p) \Rightarrow$ $s_l$ and $s'$ do not intersect below the current sweep line. \\
$s''=s_2, s_r=s_1$ \\
{\scshape{FindNewEvent}}$(s'',s_r,p) \Rightarrow$ $s''$ and $s_r$ intersect below the current sweep line in $p_{s_1,s_2}$. \\
$\hspace*{49.5mm} \Rightarrow Q.insert(p_{s_1,s_2})$ \\
$\hspace*{49.5mm} \Rightarrow Q=tree(p_{s_1,s_2},p_3,q_1,q_3,p_4,q_2,p_5,q_5,q_4)$ \\
intersection\_points $= \emptyset$ \par
\vspace*{-3mm}
\hrulefill \\
$p = p_{s_1,s_2}, U = \emptyset, L=\emptyset, C=\{s_1,s_2\}$ \\
$|L \cup U \cup C|>1 \text{ is true} \Rightarrow $ $p = p_{s_1,s_2}$ is intersection of $s_1$ and $s_2$ \\
$T.delete(s_1), \ T.delete(s_2), \ T.insert(s_1), \ T.insert(s_2) \ \Rightarrow T=tree(s_1,s_2)$ \\
$U \cup C=\{s_1, s_2\}$ \\
$s'=s_1, s_l=null$ \\
{\scshape{FindNewEvent}}$(s_l,s',p) \Rightarrow$ $s_l$ and $s'$ do not intersect below the current sweep line. \\
$s''=s_2, s_r=null$ \\
{\scshape{FindNewEvent}}$(s'',s_r,p) \Rightarrow$ $s''$ and $s_r$ do not intersect below the current sweep line. \\
$\hspace*{49.5mm} \Rightarrow Q=tree(p_3,q_1,q_3,p_4,q_2,p_5,q_5,q_4)$ \\
intersection\_points $= \{p_{s_1,s_2}\}$ \par
\vspace*{-3mm}
\hrulefill \\
\newpage
$p = p_3, U = \{s_3\}, L=\emptyset, C=\emptyset$ \\
$|L \cup U \cup C|>1 \text{ is false}$ \\
$\text{no deletion, }T.insert(s_3) \Rightarrow T=tree(s_3,s_1,s_2)$ \\
$U \cup C=\{s_3\}$ \\
$s'=s_3, s_l=null$ \\
{\scshape{FindNewEvent}}$(s_l,s',p) \Rightarrow$ $s_l$ and $s'$ do not intersect below the current sweep line. \\
$s''=s_3, s_r=s_2$ \\
{\scshape{FindNewEvent}}$(s'',s_r,p) \Rightarrow$ $s''$ and $s_r$ do not intersect below the current sweep line. \\
$\hspace*{49.5mm} \Rightarrow Q=tree(q_1,q_3,p_4,q_2,p_5,q_5,q_4)$ \\
intersection\_points $= \{p_{s_1,s_2}\}$ \par
\vspace*{-3mm}
\hrulefill \\
$p = q_1=q_3, U = \emptyset, L=\{s_1,s_3\}, C=\emptyset$ \\
$|L \cup U \cup C|>1 \text{ is true} \Rightarrow $ $p = q_1=q_3$ is intersection of $s_1$ and $s_3$ \\
$T.delete(s_1), T.delete(s_3) \text{, no insertion} \Rightarrow T=tree(s_2)$ \\
$U \cup C=\emptyset$ \\
$s_l=null, s_r=null$ \\
{\scshape{FindNewEvent}}$(s_l,s_r,p) \Rightarrow$ $s_l$ and $s_r$ do not intersect below the current sweep line. \\
$\hspace*{49.5mm} \Rightarrow Q=tree(p_4,q_2,p_5,q_5,q_4)$ \\
intersection\_points $= \{p_{s_1,s_2},q_1=q_3\}$ \par
\vspace*{-3mm}
\hrulefill \\
$p = p_4, U = \{s_4\}, L=\emptyset, C=\emptyset$ \\
$|L \cup U \cup C|>1 \text{ is false}$ \\
$\text{no deletion}, \ T.insert(s_4) \Rightarrow T=tree(s_4,s_2)$ \\
$U \cup C=\{s_4\}$ \\
$s'=s_4, s_l=null$ \\
{\scshape{FindNewEvent}}$(s_l,s',p) \Rightarrow$ $s_l$ and $s'$ do not intersect below the current sweep line. \\
$s''=s_4, s_r=s_2$ \\
{\scshape{FindNewEvent}}$(s'',s_r,p) \Rightarrow$ $s''$ and $s_r$ do not intersect below the current sweep line. \\
$\hspace*{49.5mm} \Rightarrow Q=tree(q_2,p_5,q_5,q_4)$ \\
intersection\_points $= \{p_{s_1,s_2},q_1=q_3\}$ \par
\vspace*{-3mm}
\hrulefill \\
$p = q_2=p_5, U = \{s_5\}, L=\{s_2\}, C=\emptyset$ \\
$|L \cup U \cup C|>1 \text{ is true} \Rightarrow $ $p = q_2=p_5$ is intersection of $s_2$ and $s_5$ \\
$T.delete(s_2), T.insert(s_5) \Rightarrow T=tree(s_4,s_5)$ \\
$U \cup C=\{s_5\}$ \\
$s'=s_5, s_l=s_4$ \\
{\scshape{FindNewEvent}}$(s_l,s',p) \Rightarrow$ $s_l$ and $s'$ do not intersect below the current sweep line. \\
$s''=s_5, s_r=null$ \\
{\scshape{FindNewEvent}}$(s'',s_r,p) \Rightarrow$ $s''$ and $s_r$ do not intersect below the current sweep line. \\
$\hspace*{49.5mm} \Rightarrow Q=tree(q_5,q_4)$ \\
intersection\_points $= \{p_{s_1,s_2},q_1=q_3,q_2=p_5\}$ \par
\vspace*{-3mm}
\hrulefill \\
\newpage
$p = q_5, U = \emptyset, L=\{s_5\}, C=\emptyset$ \\
$|L \cup U \cup C|>1 \text{ is false}$ \\
$T.delete(s_5), \text{no insertion} \Rightarrow T=tree(s_4)$ \\
$U \cup C=\emptyset$ \\
$s_l=null, s_r=null$ \\
{\scshape{FindNewEvent}}$(s_l,s_r,p) \Rightarrow$ $s_l$ and $s_r$ do not intersect below the current sweep line. \\
$\hspace*{49.5mm} \Rightarrow Q=tree(q_4)$ \\
intersection\_points $= \{p_{s_1,s_2},q_1=q_3,q_2=p_5\}$ \par
\vspace*{-3mm}
\hrulefill \\
$p = q_4, U = \emptyset, L=\{s_4\}, C=\emptyset$ \\
$|L \cup U \cup C|>1 \text{ is false}$ \\
$T.delete(s_4), \text{no insertion} \Rightarrow T=tree()$ \\
$U \cup C=\emptyset$ \\
$s_l=null, s_r=null$ \\
{\scshape{FindNewEvent}}$(s_l,s_r,p) \Rightarrow$ $s_l$ and $s_r$ do not intersect below the current sweep line. \\
$\hspace*{49.5mm} \Rightarrow Q=tree()$ \\
intersection\_points $= \{p_{s_1,s_2},q_1=q_3,q_2=p_5\}$ \par
\vspace*{-3mm}
\hrulefill \\
\newpage



\section*{Exercise 3 (Pyramids Skyline)}
\algrenewcommand\algorithmicrequire{\textbf{Input:}}
\algrenewcommand\algorithmicensure{\textbf{Output:}}

We use a divide-and-conquer approach to compute the pyramid skyline efficiently, which
follows the concept of merge sort. The following algorithm operates on sets of skylines
(which are themselves assumed to be sets of line segments),
so we have to convert the pyramid representations to skylines: But this is easy: if a
pyramid is represented by the top corner-coordinate $(x,y)$, then this yields the skyline
$\{ \overline{(0,y-x), (x,y)}, \overline{(x,y), (0,y+x)} \}$.

As we see in the above pyramid skyline, the assumption is that the $x$-axis is the horizon
and every pyramid has topmost corner with positive $y$-coordinate. In particular, every
skyline constructed from such pyramid skylines will start and end with a point with
coordinate $y = 0$.

Before applying the following algorithm, we need to (potentially) simplify the set of
pyramids: We want to ensure that no pyramids share a left or right side, as this will
create difficult special cases in the following computations

This is easy: Sort the set of pyramids (possible in $\bigO(n\log n)$ lexicographically
with respect to the key ($x$-coordinate of left corner, $y$-coordinate of top corner).
We can then iterate through this set and compare $x$-coordinates of the left corners of
subsequent pyramids: if they are equal, then we remove the one with lower top coordinates
as it will be fully contained in the other pyramid and thus cannot contribute to the
skyline. This can clearly be done in $\bigO(n)$. We can do the same with the right
coordinate contributing another $\bigO(n\log n)$.

\begin{algorithm}
  \caption{$\text{\textsc{Skyline}}(T)$}
  \begin{algorithmic}[1]
    \Require $T$ the set of individual skylines
    \Ensure $S$ the combined skyline
    \State $l := \mathrm{length}(T)$
    \If{$l < 2$}
      \State \Return $T$
    \EndIf
    \State Partition $T$ into two sets $T_1$, $T_2$ of (almost) same size, (e.g.\ split in
      the middle)
    \State $fst = \text{\textsc{Skyline}}(T_1)$
    \State $snd = \text{\textsc{Skyline}}(T_2)$
    \State\Return \textsc{MergeSkylines}$(fst, snd)$
  \end{algorithmic}
\end{algorithm}

This is the outer part of the algorithm, the actual computations happen in
\textsc{MergeSkylines}. We give an informal description:

Assume we merge two skylines $T_1$, $T_2$ which consist of $|T_1|=n$, $|T_2|=m$ line
segments respectively. In the following, a line segment will \textit{start} at its left end
and \textit{end} at its right end, since the algorithm will traverse both skylines from left
to right. We can assume that the skylines are sorted from left to right.

\begin{enumerate}
  \item Ensure that the first (i.e.\ left-most) line segment in $T_1$ starts at lower
    $x$-coordinate than the first one of $T_2$. If this is not the case, rename the sets
    accordingly.

    Note that, by the preprocessing, $T_1$ and $T_2$ cannot start with the same
    $x$-coordinate.
  \item If the end point of the last (right-most) line segment of $T_1$ has $x$-coordinate
    less than or equal to the $x$-coordinate of the start point of the first line segment
    of $T_1$, then there is nothing to do as the skylines are disjoint, so return
    $S=T_1\cup T_2$.
  \item Initialize the new skyline $S = \emptyset$.
  \item Move the first line segment from $T_1$ to $S$ (by the choices above it is the
    left-most and must be in the skyline)
  \item While $T_1 \neq \emptyset$ and $T_2\neq\emptyset$ do the following, where $t_1$
    and $t_2$ shall be the first line segments in $T_1$ and $T_2$ respectively.
    \begin{enumerate}
      \item If the endpoint of $t_2$ lies left of the starting point of $t_1$ (by
        construction thus far the end point of the skyline $S$), then remove $t_2$ as it
        cannot occur in the skyline and restart the loop.
      \item Otherwise check if $t_1$ and $t_2$ intersect:
        \begin{itemize}
          \item If they do intersect, then they cannot be collinear (by preprocessing) and
            must intersect in a point $x$: Then denote by $t_{1,x}$ the line segment from
            the start of $t_1$ to $x$ and by $t_{2,x}$ the line segment from $x$ to the
            end of $t_2$. But $t_{2,x}$ cannot be a point (again by preprocessing: if some
            line segments only meet in an endpoint of one, then we must also have
            collinear intersecting line segments which is impossible) and by
            construction all line segments from $T_2$ thus far must lie below the
            constructed skyline, so $t_{1,x}$ and $t_{2,x}$ form a \enquote{valley}. We do
            the following
            \begin{itemize}
              \item Add $t_{1,x}$ to $S$,
              \item Remove $t_1$ from $T_1$,
              \item Replace $t_2$ in $T_2$ with $t_{2,x}$
              \item Swap $T_1$ and $T_2$. This is necessary as such an intersection means
                we switch over from one skyline to another.
            \end{itemize}
            Then restart the loop.
          \item If they do not intersect then there are again two cases:
            \begin{itemize}
              \item The endpoint of $t_2$ lies right of the endpoint of $t_1$. Then no
                other element in $T_2$ can intersect with $t_1$ and hence $t_1$ must be
                part of the skyline. Hence move $t_1$ from $T_1$ to $T_2$ and restart the
                loop.

                Note that $t_2$ might still intersect other elements of $T_1$.
              \item If the endpoint of $t_2$ lies left of the endpoint of $t_1$, then
                $t_2$ cannot intersect any element in $T_1$. Hence $t_2$ cannot lie on the
                combined skyline, so remove it from $T_2$ and restart the loop.
            \end{itemize}
        \end{itemize}
    \end{enumerate}
  \item If elements in $T_2$ remain, they must be disjoint from the constructed skyline
    $S$ so we simply return $S \cup T_2$.
\end{enumerate}

% TODO: hand-wavy
In less formal terms, the above algorithm \enquote{walks along} $T_1$ until it intersects
$T_2$, then continues on $T_2$ and so on until the skyline is complete. Further, for each
iteration of the while loop (5.) we discard one element of $T_1$ or $T_2$ and all the
checks we need to do are clearly possible in constant time, so \textsc{MergeSkylines} has
runtime in $\bigO(m+n) = \bigO(|T_1|+|T_2|)$. Also, each line segment in $T_1$ or $T_2$
can intersect at most one line segment from the other set by the structure of skylines, so
the amount of line segments in the new skyline can be at most $m+n$ as well. Hence, if $T$
is the complexity function for \textsc{Skyline}, then we obtain the following recurrence
relation
\begin{align*}
  T(n) &= 2 T\left(\frac n2\right) + \bigO(\text{\textsc{MergeSkylines}}) \\
       &= 2 T\left(\frac n2\right) + \bigO\left(\frac n2 + \frac n2\right) \\
       &= 2 T\left(\frac n2\right) + \bigO(n)
\end{align*}
By the Master Theorem we now know that $T \in \bigO(n\log n)$, so, together with the
preprocessing, the algorithm has a total complexity of $\bigO(n\log n)$.

\newpage

\end{document}



